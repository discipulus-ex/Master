\documentclass[a4paper]{article}
\usepackage[english]{babel}
\usepackage[utf8]{inputenc}
\usepackage{hyperref}
\usepackage{graphicx}
\usepackage{array}
\usepackage{pifont}
\usepackage{threeparttable}
\usepackage{amsmath}
\usepackage{booktabs}
\usepackage{lipsum}

\renewcommand{\arraystretch}{1.5}

\graphicspath{ {./images/} }
\bibliographystyle{plain}

\title{\Huge{Prosthetic Hand} } 
\author{Dylan Duunk \\ \textit{Discipulus-Ex} }
\date{\today}

\begin{document}

\begin{titlepage}
    \maketitle
\end{titlepage}

\pagenumbering{roman}

\newpage

\begin{abstract}
    \noindent 
    \lipsum[1]
\end{abstract}

\newpage
\tableofcontents
\newpage
\pagenumbering{arabic}

\section{Introduction}
\lipsum[1]

\section{Materials}
Material choice plays a major role in prostheses.
Prostheses must be both strong and lightweight, they must be able to withstand the extensive use of the owner without obstructing there range of motion.
\\
Below I will review a variety of materials for our prosthetic hand.

\subsection{Plastic}
PLA and ABS are 2 of the most common FDM (Fused deposition modeling) desktop printing materials.
Both materials are thermoplastics, meaning they become pliable or moldable at a certain elevated temperature and solidifies upon cooling.
Via the FDM process, both materials are melted and then extruded trough a nozzle to build up the layers that create a final part.
\\ \\
Table \ref{tab:pla-abs} below compares the main properties between PLA \& ABS:
\begin{table}[ht]
    \centering
    \begin{threeparttable}
        \begin{tabular}[t]{>{\bfseries}l l l}
            \toprule
            Properties\tnote{1} & \textbf{PLA} & \textbf{ABS} \\
            \midrule
            Density & $1.3 g/cm^3$ & $1.0 - 1.4 g/cm^3$ \\
            Elongation & 6\% & 3.5-50\% \\ %https://en.wikipedia.org/wiki/Deformation_(mechanics)#Stretch_ratio
            Flexural Modulus & 4GPa & 2.1-7.6 GPa \\ %https://en.wikipedia.org/wiki/Flexural_modulus
            Melting Point & $160\,^{\circ}{\rm C}$ & N/A (amorphous) \\
            Biodegradable & Yes, under the correct conditions & No \\
            Glass Transition Temperature & $60\,^{\circ}{\rm C}$ & $105\,^{\circ}{\rm C}$ \\ %https://en.wikipedia.org/wiki/Glass_transition
            \bottomrule
        \end{tabular}
        \caption{Comparing PLA with ABS}
        \label{tab:pla-abs}
        \begin{tablenotes}
            \item[1] \textit{Sourced from MakeItFrom \cite{MakeItFrom}}
        \end{tablenotes}    
    \end{threeparttable}    
\end{table}

\subsection{Silicone}
Silicone is a dynamic material that moves with the body while simultaneously offering 
an enhanced grip on the residual limb and improved suspension of the prosthesis.
Silicone is mostly used in realistic looking prosthetics and for mold making.
\\ \\ 
Table \ref{tab:silicone-properties} below gives the main properties of silicone:
\begin{table}[ht]
    \centering
    \begin{threeparttable}
        \begin{tabular}[t]{>{\bfseries}l l}
            \toprule
            Properties\tnote{1} & \textbf{Silicone plastic}  \\
            \midrule
            Density & $1.9 g/cm^3$  \\
            Elastic Modulus & 9.0 GPa \\ %https://en.wikipedia.org/wiki/Elastic_modulus
            Max. Temperature: Decomposition & $480\,^{\circ}{\rm C}$  \\
            Biodegradable & No  \\
            Glass Transition Temperature & $200\,^{\circ}{\rm C}$  \\ %https://en.wikipedia.org/wiki/Glass_transition
            \bottomrule
        \end{tabular}
        \caption{Properties silicone plastic}
        \label{tab:silicone-properties}
        \begin{tablenotes}
            \item[1] \textit{Sourced from MakeItFrom \cite{MakeItFrom}}
        \end{tablenotes}    
    \end{threeparttable}    
\end{table}

\subsection{Carbon Fiber}
Carbon fiber reinforced plastic (CFRP), is an extremely strong and light fiber-reinforced plastic which contains carbon fibers.
Carbon fiber is often used in prosthetics, sports equipment, aerospace and wherever the high strength-to-weight ratio and stiffness from carbon fiber are required.
CFRPs are composite materials.
In this case the composite consists of two parts: a matrix and a reinforcement.
In this case the reinforcement is carbon fiber, which provides the strength.
The matix is usually a polymer resin, such as epoxy, to bind the reinforcements together.
This makes the material properties depend on those distinct elements.

\subsection{Metal}
Alloys containing titanium are known for their high strength, lightweight, and exceptional corrosion resistance.
Despite being as strong as steel, titanium is about 40\% lighter in weight.
Titanium is also formidable in its resistance to corrosion by both water and chemical media.
\\
Because titanium has a low modulus of elasticity that means titanium is not also very flexible, but returns to its original shape after bending.
\\ \\
Aluminium is a very light metal with a specific weight of $2.7 g/cm^3$, about a third that of steel.
Its strength can be adapted to the apllication required by modifying the composition of its alloys.
Aluminium naturally generates a protective oxide coating and is highly corrosion resistant which can be further imporeved by different types of surface treatments. 
This is particularly usefull for applications where protection and conservation are required.

\subsection{Morphological chart}
% 0 stars 
% \ding{73}\ding{73}\ding{73}\ding{73}\ding{73}
% 1 stars 
% \ding{72}\ding{73}\ding{73}\ding{73}\ding{73} 
% 2 stars 
% \ding{72}\ding{72}\ding{73}\ding{73}\ding{73}
% 3 stars 
% \ding{72}\ding{72}\ding{72}\ding{73}\ding{73}
% 4 stars 
% \ding{72}\ding{72}\ding{72}\ding{72}\ding{73}
% 5 stars 
% \ding{72}\ding{72}\ding{72}\ding{72}\ding{72} 
\begin{table}[ht]
    \centering
    \begin{threeparttable}
        \begin{tabular}[t]{>{\bfseries}l c c c c c c}
            \toprule
            & \textbf{PLA} & \textbf{ABS} & \textbf{Silicone} & \textbf{Carbon} & \textbf{Titanium\tnote{1}} & \textbf{Aluminium\tnote{1}}  \\
            \midrule
            Easy of use & \ding{72}\ding{72}\ding{72}\ding{72}\ding{72} & \ding{72}\ding{72}\ding{72}\ding{72}\ding{73} & \ding{72}\ding{72}\ding{72}\ding{73}\ding{73} & \ding{72}\ding{72}\ding{73}\ding{73}\ding{73} & \ding{72}\ding{73}\ding{73}\ding{73}\ding{73} & \ding{72}\ding{72}\ding{73}\ding{73}\ding{73} \\
            Strength & \ding{72}\ding{72}\ding{73}\ding{73}\ding{73} & \ding{72}\ding{72}\ding{72}\ding{73}\ding{73} & \ding{72}\ding{72}\ding{72}\ding{72}\ding{73} & \ding{72}\ding{72}\ding{72}\ding{72}\ding{72} & \ding{72}\ding{72}\ding{72}\ding{72}\ding{72} & \ding{72}\ding{72}\ding{72}\ding{72}\ding{72} \\
            Weight\tnote{2} & \ding{72}\ding{72}\ding{72}\ding{72}\ding{72} & \ding{72}\ding{72}\ding{72}\ding{72}\ding{72}  & \ding{72}\ding{72}\ding{72}\ding{72}\ding{72} & \ding{72}\ding{72}\ding{72}\ding{72}\ding{72} & \ding{72}\ding{72}\ding{72}\ding{72}\ding{72} & \ding{72}\ding{72}\ding{72}\ding{72}\ding{72}  \\
            Elastic Modulus & \ding{72}\ding{72}\ding{73}\ding{73}\ding{73} & \ding{72}\ding{72}\ding{72}\ding{73}\ding{73} & \ding{72}\ding{72}\ding{72}\ding{72}\ding{72} & \ding{72}\ding{72}\ding{72}\ding{73}\ding{73} & \ding{72}\ding{72}\ding{72}\ding{72}\ding{72} & \ding{72}\ding{72}\ding{72}\ding{73}\ding{73} \\
            Price & \ding{72}\ding{72}\ding{72}\ding{72}\ding{72} & \ding{72}\ding{72}\ding{72}\ding{72}\ding{72} & \ding{72}\ding{72}\ding{72}\ding{72}\ding{73} & \ding{72}\ding{72}\ding{73}\ding{73}\ding{73} & \ding{72}\ding{73}\ding{73}\ding{73}\ding{73} & \ding{72}\ding{72}\ding{72}\ding{73}\ding{73} \\
            Usability\tnote{3} & \ding{72}\ding{72}\ding{72}\ding{72}\ding{72} & \ding{72}\ding{72}\ding{72}\ding{72}\ding{73} & \ding{72}\ding{72}\ding{72}\ding{73}\ding{73} & \ding{72}\ding{73}\ding{73}\ding{73}\ding{73} & \ding{73}\ding{73}\ding{73}\ding{73}\ding{73} & \ding{72}\ding{73}\ding{73}\ding{73}\ding{73} \\
            \bottomrule
        \end{tabular}
        \caption{Materials Morphological Chart}
        \label{tab:morphchart-material}  
        \begin{tablenotes}
            \item[] These values in this Morphological Chart are based on eachother.
            \item[1] The material properties of these materials can vary based on the alloys used.
            \item[2] Weight is based on how lightweight the material is.
            \item[3] Usability in this case is based on how usefull this material will be for our prototype.  
        \end{tablenotes}    
    \end{threeparttable}   
\end{table}

\subsection{Conclusion}
A combination of PLA, ABS and Silicone would be the best choice material wise. 
This is because PLA and ABS are easy to use for rapid prototyping, in a relatively short time we can put together a functional prototype.
When further developing this prototype to a product i suggest to look into replacing parts with stronger materials.
When it comes to silicone, silicone is very suitable for the grip pads.
These pads make it easier to pick up objects.

\newpage

\bibliography{literature}

\end{document}