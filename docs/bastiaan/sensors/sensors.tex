\documentclass[11pt,a4paper]{article}

\usepackage[utf8]{inputenc}
\usepackage[T1]{fontenc}
\usepackage[english]{babel}

\usepackage[margin=1.0in]{geometry}
\usepackage[parfill]{parskip}
\tolerance=1000
\hyphenpenalty=1000
%\hbadness=10000
%\widowpenalties=1 10000
%\raggedbottom

\usepackage[bookmarks]{hyperref}
\usepackage[defernumbers,sortcites,sorting=none]{biblatex}
%\usepackage{blindtext}
\usepackage{booktabs}
\usepackage{pifont}
%\usepackage{enumitem}
%\usepackage{graphicx, wrapfig}
%\usepackage{mathtools}
%\usepackage{subcaption}
%\usepackage{textcomp}

\usepackage{cochineal}
\usepackage[varqu,varl,var0]{zi4} % Inconsolata
\usepackage[scale=.95,type1]{cabin}
\usepackage[cochineal,bigdelims,cmintegrals,vvarbb]{newtxmath}
\usepackage[cal=boondoxo]{mathalfa}

\title{Research on sensors}
\author{Bastiaan Teeuwen (0945334) --- Discipulus Ex}
\date{\today}

\addbibresource{bibliography.bib}

\begin{document}

\maketitle

In this document, I will review a variety of sensors that we could use to get
input for our prosthetic hand.

\section{Muscle sensors}
One possibility would be to use muscle sensors, also called electromyography
(EMG), to record electrical activity produced by the muscles. Electrodes would
be placed on one's upper arm or forearm. An EMG device would be used to detect
the level of activation of certain muscles.~\cite{emg}

A quick Internet search reveals that muscle sensors are relatively easy to
acquire and interface with. Take the
\href{https://www.sparkfun.com/products/13723}{MyoWare Muscle Sensor} for
example. The only interfacing device required is an Arduino to read out the
voltage levels.

Various studies have been done on using EMGs to capture finger movement. In one
study, four electrodes were placed on the forearm after which various finger
motions were performed. The data was fed into and EMG and after a 25 training
sessions the finger motions could be precisely registered. The results are
pretty stunning with a reported accuracy of 97.75\%.~\cite{decoding}\\
The question is whether this will still work when the patient is missing his or
her fingers.

\section{Brainwave sensors}
A brainwave sensor, also called electroencephalography (EEG) is a method of
monitoring electrical brainwave activity. The idea is that electrodes are placed
on one's skull after which the EEG device will record the various known brain
waves.~\cite{eeg} Let's look at the various types of brainwaves to see if
they're of any use to us.

EEG devices are relatively more expensive and not as easy to interface with as
EMGs.~\cite{eegbuy}

\subsection{Delta wave}
First, we have the $\delta$ (delta) wave, which is commonly used to analyse
one's sleep pattern.~\cite{delta}

\subsection{Theta wave}
Then we have the $\theta$ (theta) wave, which can be linked to whether one is in
an idle state or performing a certain task. It is also related to motor behavior
as a result of sensory stimuli.~\cite{theta}

\subsection{Alpha wave}
The $\alpha$ (alpha) wave, just like the Delta wave, is linked to sleep activity
as well. It can also be weakly linked to the Visual cortex which is a part of
the brain that processes visual information.~\cite{alpha}

\subsection{Beta wave}
The $\beta$ (beta) wave is associated with various different state of
consciousness like flow, relaxation, anxiety, and so on.~\cite{flow}
Furthermore, this brain wave is also associated with isotonic (static) muscle
contractions, meaning the contraction of muscles without movement (like when
performing pushups or lat pulldowns~\cite{isotonic}).~\cite{beta}

\subsection{Gamma wave}
No much is known about the $\gamma$ (gamma) wave, other than that it is possibly
related to consciousness.~\cite{gamma}

\subsection{Mu wave}
The final distinguishable brain wave is the $\mu$ (mu) wave. The Mu wave is most
prominent in the Motor cortex, a part of the brain which controls coordinated
movements. Besides actual motor movement, intention of movement can also be
traced back to the Mu wave. This is especially of interest to people who are
missing a certain body part. One would be able to visualize a movement and this
could (with enough training) be registered as a Mu wave on an EEG
device.~\cite{mu}

\subsection{Conclusion}
The $\theta$ (theta) wave could possibly be of interest to us as it is related
to motor movement. But the $\mu$ (mu) wave would probably be of greater interest
to us as it can be strongly linked to motor activity and intention of motor
activity regardless of stimuli.
We could look for a pattern in both of these waves relating to the muscle
movement in the hand or fingers.

\section{Morphological chart}

\begin{center}
\begin{tabular}{l c c}
	& EMG & EEG\\
	\midrule
	Ease of use & \ding{72}\ding{72}\ding{72}\ding{72}\ding{73} & \ding{72}\ding{72}\ding{73}\ding{73}\ding{73}\\
	Observability & \ding{72}\ding{72}\ding{72}\ding{72}\ding{73} & \ding{72}\ding{72}\ding{72}\ding{73}\ding{73}\\
	Intention observability & \ding{73}\ding{73}\ding{73}\ding{73}\ding{73} & \ding{72}\ding{72}\ding{72}\ding{72}\ding{73}\\
	Research & \ding{72}\ding{72}\ding{72}\ding{72}\ding{73} & \ding{72}\ding{72}\ding{72}\ding{73}\ding{73}\\
	Interfacing & \ding{72}\ding{72}\ding{72}\ding{72}\ding{73} & \ding{72}\ding{72}\ding{72}\ding{73}\ding{73}\\
	Price & \ding{72}\ding{72}\ding{72}\ding{72}\ding{73} & \ding{72}\ding{72}\ding{73}\ding{73}\ding{73}\\
	\bottomrule
\end{tabular}
\end{center}

\section{Conclusion}
An advantage of using electroencephalography over electromyography is that the
brainwaves for controlling a body part are always detectable even if a patient
is missing that body part. The advantage of using an EMG is that they're
easier to work with as less decoding is required and less noise is present.
Furthermore, more research has been done on using EMGs for prosthetic
applications.~\cite{emg}

I would recommend we try the EMG first and only try using the EEG if we are
unable to register movement in patients with missing fingers.

\newpage
\printbibliography{}

\end{document}
